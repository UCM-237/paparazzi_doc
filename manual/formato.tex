\usepackage[pdftex]{graphicx}
\usepackage{epstopdf}
\usepackage{subfigure}
\usepackage{amsmath,amsthm}
\usepackage{tikz}
\usepackage{circuitikz}
\usetikzlibrary{babel}
\usetikzlibrary{shapes, arrows, patterns, angles, quotes}
\textwidth= 15cm
\evensidemargin=0cm
\usepackage[spanish]{babel}
\usepackage[utf8]{inputenc}
\usepackage{textcomp}
\usepackage{amstext}
\usepackage{amsfonts}
\usepackage{amssymb}
\usepackage{comment}
\usepackage[hyperindex=true,breaklinks=true,colorlinks=true,linkcolor=blue]{hyperref}
\renewcommand{\tablename}{Tabla}
\renewcommand{\listtablename}{\'Indice de Tablas}
\usepackage{color}
\definecolor{gray97}{gray}{.97}
\definecolor{gray75}{gray}{.75}
\definecolor{gray45}{gray}{.45}
\usepackage{color} %red, green, blue, yellow, cyan, magenta, black, white
\definecolor{mygreen}{RGB}{28,172,0} % color values Red, Green, Blue
\definecolor{mylilas}{RGB}{170,55,241}

\usepackage{listings}
\lstloadlanguages{C,XML}
\lstset{ frame=Ltb,
	framerule=0pt,
	aboveskip=0.5cm,
	framextopmargin=3pt,
	framexbottommargin=3pt,
	framexleftmargin=0.8cm,
	framesep=0pt,
	rulesep=.4pt,
	backgroundcolor=\color{gray97},
	rulesepcolor=\color{black},
	%
	stringstyle=\ttfamily,
	showstringspaces = false,
	basicstyle=\small\ttfamily,
	commentstyle=\color{gray45},
	keywordstyle=\bfseries,
	%
	numbers=left,
	numbersep=15pt,
	numberstyle=\tiny,
	numberfirstline = false,
	breaklines=true,
}

% minimizar fragmentado de listados
\lstnewenvironment{listing}[1][]
{\lstset{#1}\pagebreak[0]}{\pagebreak[0]}

\lstdefinestyle{consola}
{basicstyle=\scriptsize\bf\ttfamily,
	backgroundcolor=\color{gray75},
}

\lstdefinestyle{C}
{language=C,
}

\lstdefinestyle{XML}
{language=XML,
}


\usepackage{multirow}
\usepackage{makeidx}

%\usepackage{draftwatermark}
%\SetWatermarkText{Borrador,juan.jimenez@fis.ucm.es}
%\SetWatermarkScale{2}

% Atajos para el tikz
\tikzstyle{block} = [draw, rectangle, minimum width=6em]
\tikzstyle{sum} = [draw, fill=blue!20, circle, node distance=1cm]
\tikzstyle{input} = [coordinate]
\tikzstyle{output} = [coordinate]
\tikzstyle{pinstyle} = [pin edge={to-,thin,black}]

% Entornos para los "teoremas"
\newtheorem{algo}{Algoritmo}[section]
\newtheorem{theorem}{Teorema}[section]
\newtheorem{problem}{Problema}[section]
\newtheorem{corollary}{Corolario}[section]
\newtheorem{lemmas}{Lema}[section]
\newcommand*{\lema}{Lema}
\newenvironment{lemma}[1][\lema]{\begin{lemmas}[#1]\renewcommand*{\qedsymbol}{\(\Diamond\)}}{\end{lemmas}}

\theoremstyle{definition}
%\newtheorem{definition}{Definición}[section]
\newtheorem{definitions}{Definición}[section]
\newcommand*{\definicion}{Definición}
\newenvironment{definition}[1][\definicion]{\begin{definitions}[#1]\renewcommand*{\qedsymbol}{\(\bigtriangledown\)}}{\end{definitions}}

\newtheorem{examples}{Ejemplo}[section]
\newcommand*{\ejemplo}{Ejemplo}
\newenvironment{example}[1][\ejemplo]{\begin{examples}[#1]\renewcommand*{\qedsymbol}{\(\maltese\)}}{\end{examples}}
%\renewcommand{\qedsymbol}{\maltese}

\theoremstyle{remark}
\newtheorem{remark}{Atención}[section]

\lstset{language=Matlab,%
    %basicstyle=\color{red},
    breaklines=true,%
    morekeywords={matlab2tikz},
    keywordstyle=\color{blue},%
    morekeywords=[2]{1}, keywordstyle=[2]{\color{black}},
    identifierstyle=\color{black},%
    stringstyle=\color{mylilas},
    commentstyle=\color{mygreen},%
    showstringspaces=false,%without this there will be a symbol in the places where there is a space
    numbers=left,%
    numberstyle={\tiny \color{black}},% size of the numbers
    numbersep=9pt, % this defines how far the numbers are from the text
    emph=[1]{for,end,break},emphstyle=[1]\color{red}, %some words to emphasise
    %emph=[2]{word1,word2}, emphstyle=[2]{style},    
}

\graphicspath{{./figuras/}}