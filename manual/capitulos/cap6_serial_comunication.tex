\chapter{Módulo de comunicación serie}

El módulo de comunicación serie permite que Paparazzi establezca comunicación bidireccional con otro dispositivo a través de uno de los puertos serie. Con este módulo implementaremos el protocolo de comunicaciones entre paparazzi y la Raspberry (o el companion computer) correspondiente.

\section{Protocolo de comunicación}\label{sec:1}

La comunicación entre Paparazzi y el (companion Computer) CC será bidireccional. Todos los mensajes llevarán una marca de tiempo. Los mensajes establecidos son los siguientes:
\begin{enumerate}
	\item Mensajes desde Paparazzi al CC
	\begin{enumerate}
		\item Mensaje con los datos de telemetría. 
		\item Mensaje indicando al CC que puede proceder a realizar las medidas.
		\item Mensaje solicitando al CC la posición de la sonda.
	\end{enumerate}
	\item Mensajes desde el CC al Paparazzi
	\begin{enumerate}
			\item Mensaje de respuesta a la solicitud de medida.
			\item Mensaje periódico con la profundidad de la sonda (durante la medida).
			\item Mensaje de finalización de medida.
			\item Mensaje de respuesta a la solicitud de posición de la sonda con la posición.
	\end{enumerate}	
\end{enumerate}

\subsection{Configuración del puerto serie}

La configuración del puerto serie será la estándar con velocidad de $9600$ baudios, tabla \ref{tab:1}.
\begin{table}[h]
	\centering
	\caption{Configuración del puerto serie}
\begin{tabular}{|c|c|}
	\hline
	\textbf{Parámetro} & \textbf{Valor} \\ \hline \hline
	Velocidad  & $9600$ (baudios) \\\hline 
	Paridad & ninguna  \\ \hline
	Bits & $8$ \\ \hline 
	Bits de parada & $1$  \\ \hline
	Control de flujo & ninguno \\ \hline
\end{tabular}
\label{tab:1}
\end{table}


\subsection{Comunicación desde paparazzi}

Paparazzi enviará tres mensajes diferentes al CC. Uno de ellos periódico, el mensaje de la telemetría y el resto puntualmente.

Todos los mensajes comienzan con un byte que contiene el código ASCII de la letra \textbf{P} indicando que dicho mensaje procede de paparazzi. Todos los mensajes terminan con $2$ bytes que corresponden al checksum.

Las cantidades numéricas enteras sin signo se codifican usando el formato Little Endian (el byte más significativo ocupa la posición de menor índice en el mensaje). Por ejemplo si queremos enviar el número decimal $72$ como hexadecimal de $2$ bytes lo codificaremos como: $\{0x00,0x48\}$.

Las cantidades numéricas con signo se codifican de igual manera que las enteras sin signo pero el byte de menor índice se usa para indicar el signo: $0x01$ si el signo es negativo y $0x00$ si es positivo. Por ejemplo si usamos $3$ bytes para codificar el número $-72$ lo codificaremos como $\{0x01,0x00,0x48\}$ y el número $72$ será $\{0x00,0x00,0x48\}$.

\subsubsection{Mensaje de telemetría}

El mensaje de telemetría es un mensaje periódico que envía paparazzi al CC con los datos de telemetría: GPS y medida de Sonar. Tiene una longitud de $25$ bytes y la estructura en bytes especificada en la tabla \ref{tab:2}.

\begin{table}[h]
	\centering
	\caption{Estructura del mensaje de telemetría}
	\begin{tabular}{|c|c|c|c|}\hline 
		\textbf{Posición}	& \textbf{Valor} & \textbf{Tipo} &\textbf{Número de bytes} \\ \hline \hline 
		$0$		& Byte de sincronía ``P''				& ASCII	 			&	$1$ \\  \hline
		$1$		& Tipo de mensaje ``T''				& ASCII	 			&	$1$ \\  \hline
		$2-3$	& Marca de tiempo (s)				& Entero sin signo	&   $2$ \\  \hline
		$4-8$	& Longitud ($grad \cdot 10^{7}$)	& Entero con signo	&   $5$ \\  \hline
		$9-13$	& Latitud ($grad \cdot 10^{7}$)		& Entero con signo	&  	$5$ \\  \hline
		$14-17$	& Altitud ($mm$)					& Entero sin signo	&   $4$ \\  \hline
		$18-21$	& Distancia sonar ($mm$)			& Entero sin signo	&   $4$ \\  \hline
		$22$	& Confianza sonar ($\%$)			& Entero sin signo	&   $1$ \\  \hline
		$23-24$	& Checksum 							& Entero sin signo	&   $2$ \\  \hline
	\end{tabular}
	\label{tab:2}
\end{table}

En la tabla \ref{tab:3} puede verse un ejemplo de uno de estos mensajes.

\begin{table}
	\centering
	\caption{Ejemplo de un mensaje de telemetría}
	\begin{tabular}{|c|c|c|c|}\hline
		\textbf{Byte} 	&	\textbf{Valor (en hexadecimal)}	&\textbf{Valor}	&\textbf{Significado} \\ \hline \hline
		$0$ 			&  $0x50$			& ``P''	& Byte sincronía	\\ \hline
		$1$				&  $0x54$			& ``T''	& Tipo mensaje		\\ \hline
		$2$				&  $0x00$			& \multirow{2}{*}{$72$ segundos} & \multirow{2}{*}{Tiempo} \\
		$3$				&  $0x48$			&  & \\ \hline	
		$4$				&  $0x01$			&  \multirow{5}{*}{$-37260579$ ($grad \cdot 10^{7}$) } &  \multirow{5}{*}{Longitud}  \\
		$5$				&  $0x02$			&  &   \\ 	
		$6$				&  $0x38$			&  &    \\ 	
		$7$				&  $0x8D$			&  &    \\ 	
		$8$				&  $0x23$			&   &   \\ \hline
		$9$				&  $0x00$			& \multirow{5}{*}{$404506308$ ($grad \cdot 10^{7}$) } &  \multirow{5}{*}{Latitud}   \\
		$10$				&  $0x18$			&  &     \\ 	
		$11$				&  $0x1C$			&  &      \\ 	
		$12$				&  $0x46$			&  &      \\ 	
		$13$				&  $0xC4$			&  &      \\ \hline
		$14$				&  $0x00$			& \multirow{4}{*}{$702385$ (mm) } & \multirow{4}{*}{Altitud}\\
		$15$				&  $0x0A$			&   &     \\ 	
		$16$				&  $0xB7$			&   &         \\ 	
		$17$				&  $0xB1$			&   &        \\ \hline
		$18$				&  $0x00$			& \multirow{4}{*}{$8219$ (mm) } & \multirow{4}{*}{Distancia sonar}\\
		$19$				&  $0x00$			&     &     \\ 	
		$20$				&  $0x20$			&     &     \\ 	
		$21$				&  $0x1B$			&     &     \\ 	\hline
		$22$				&  $0x64$			& $100 \%$ &     Confianza \\ \hline
		$23$				&  $0x05$			&  \multirow{2}{*}{6405}	& \multirow{2}{*}{checksum} \\
		$24$				&  $0x19$			&     &     \\ \hline	
		
		
	\end{tabular}
	\label{tab:3}
\end{table}

\subsubsection{Mensaje indicando al CC que puede proceder a realizar las medidas}

El mensaje de inicio de medidas es un mensaje puntual que envía paparazzi al CC indicando que puede comenzar a descender la sona para tomar medidas. Incluye los datos de telemetría: GPS y medida de Sonar. Tiene una longitud de $25$ bytes y la estructura en bytes especificada en la tabla \ref{tab:4}.

\begin{table}[h]
	\centering
	\caption{Estructura del mensaje de inicio de medida}
	\begin{tabular}{|c|c|c|c|}\hline 
		\textbf{Posición}	& \textbf{Valor} & \textbf{Tipo} &\textbf{Número de bytes} \\ \hline \hline 
		$0$		& Byte de sincronía ``P''				& ASCII	 			&	$1$ \\  \hline
		$1$		& Tipo de mensaje ``M''				& ASCII	 			&	$1$ \\  \hline
		$2-3$	& Marca de tiempo (s)				& Entero sin signo	&   $2$ \\  \hline
		$4-8$	& Longitud ($grad \cdot 10^{7}$)	& Entero con signo	&   $5$ \\  \hline
		$9-13$	& Latitud ($grad \cdot 10^{7}$)		& Entero con signo	&  	$5$ \\  \hline
		$14-17$	& Altitud ($mm$)					& Entero sin signo	&   $4$ \\  \hline
		$18-21$	& Distancia sonar ($mm$)			& Entero sin signo	&   $4$ \\  \hline
		$22$	& Confianza sonar ($\%$)			& Entero sin signo	&   $1$ \\  \hline
		$23-24$	& Checksum 							& Entero sin signo	&   $2$ \\  \hline
	\end{tabular}
	\label{tab:4}
\end{table}

En la tabla \ref{tab:5} puede verse un ejemplo de un mensaje de inicio de medida.

\begin{table}
	\centering
	\caption{Ejemplo de un mensaje de inicio de medida}
	\begin{tabular}{|c|c|c|c|}\hline
		\textbf{Byte} 	&	\textbf{Valor (en hexadecimal)}	&\textbf{Valor}	&\textbf{Significado} \\ \hline \hline
		$0$ 			&  $0x50$			& ``P''	& Byte sincronía	\\ \hline
		$1$				&  $0x4D$			& ``M''	& Tipo mensaje		\\ \hline
		$2$				&  $0x00$			& \multirow{2}{*}{$72$ segundos} & \multirow{2}{*}{Tiempo} \\
		$3$				&  $0x48$			&  & \\ \hline	
		$4$				&  $0x01$			&  \multirow{5}{*}{$-37260579$ ($grad \cdot 10^{7}$) } &  \multirow{5}{*}{Longitud}  \\
		$5$				&  $0x02$			&  &   \\ 	
		$6$				&  $0x38$			&  &    \\ 	
		$7$				&  $0x8D$			&  &    \\ 	
		$8$				&  $0x23$			&   &   \\ \hline
		$9$				&  $0x00$			& \multirow{5}{*}{$404506308$ ($grad \cdot 10^{7}$) } &  \multirow{5}{*}{Latitud}   \\
		$10$				&  $0x18$			&  &     \\ 	
		$11$				&  $0x1C$			&  &      \\ 	
		$12$				&  $0x46$			&  &      \\ 	
		$13$				&  $0xC4$			&  &      \\ \hline
		$14$				&  $0x00$			& \multirow{4}{*}{$702385$ (mm) } & \multirow{4}{*}{Altitud}\\
		$15$				&  $0x0A$			&   &     \\ 	
		$16$				&  $0xB7$			&   &         \\ 	
		$17$				&  $0xB1$			&   &        \\ \hline
		$18$				&  $0x00$			& \multirow{4}{*}{$8219$ (mm) } & \multirow{4}{*}{Distancia sonar}\\
		$19$				&  $0x00$			&     &     \\ 	
		$20$				&  $0x20$			&     &     \\ 	
		$21$				&  $0x1B$			&     &     \\ 	\hline
		$22$				&  $0x64$			& $100 \%$ &     Confianza \\ \hline
		$23$				&  $0x05$			&  \multirow{2}{*}{1311}	& \multirow{2}{*}{checksum} \\
		$24$				&  $0x1F$			&     &     \\ \hline	
		
		
	\end{tabular}
	\label{tab:5}
\end{table}

\subsubsection{Mensaje solicitando al CC la posición de la sonda}

El mensaje de solicitud de posición de la sonda es un mensaje puntual que envía paparazzi al CC solicitando la posición de la sonda. Tiene una longitud de $6$ bytes y la estructura en bytes especificada en la tabla \ref{tab6}.

\begin{table}[h]
	\centering
	\caption{Estructura del mensaje de solicitud de posición de la sonda}
	\begin{tabular}{|c|c|c|c|}\hline 
		\textbf{Posición}	& \textbf{Valor} & \textbf{Tipo} &\textbf{Número de bytes} \\ \hline \hline 
		$0$		& Byte de sincronía ``P''				& ASCII	 			&	$1$ \\  \hline
		$1$		& Tipo de mensaje ``S''				& ASCII	 			&	$1$ \\  \hline
		$2-3$	& Marca de tiempo (s)				& Entero sin signo	&   $2$ \\  \hline
		$4-5$	& Checksum 							& Entero sin signo	&   $2$ \\  \hline
	\end{tabular}
	\label{tab6}
\end{table}

En la tabla \ref{tab7} puede verse un ejemplo de un mensaje de solicitud de posición de la sonda.

\begin{table}[h]
	\centering
	\caption{Ejemplo de solicitud de posición de la sonda}
	\begin{tabular}{|c|c|c|c|}\hline
		\textbf{Byte} 	&	\textbf{Valor (en hexadecimal)}	&\textbf{Valor}	&\textbf{Significado} \\ \hline \hline
		$0$ 			&  $0x50$			& ``P''	& Byte sincronía	\\ \hline
		$1$				&  $0x53$			& ``S''	& Tipo mensaje		\\ \hline
		$2$				&  $0x00$			& \multirow{2}{*}{$72$ segundos} & \multirow{2}{*}{Tiempo} \\
		$3$				&  $0x48$			&  & \\ \hline	
		$4$				&  $0x00$			&  \multirow{2}{*}{235}	& \multirow{2}{*}{checksum} \\
		$5$				&  $0xEB$			&     &     \\ \hline	
		
		
	\end{tabular}
	\label{tab7}
\end{table}

\subsection{Comunicación desde el companion computer}

El CC enviará a paparazzi cuatro mensajes diferentes. Uno de esos mensajes es periódico y se enviará cada segundo mientras la sonda está midiendo. El resto son mensajes puntuales. 

Todos los mensajes comienzan con un byte que contiene el código ASCII de la letra \textbf{R} indicando que dicho mensaje procede de la Raspberry. Todos los mensajes terminan con $2$ bytes que corresponden al checksum.

Las cantidades numéricas se codifican de la misma manera que en el caso de la comunicación desde paparazzi al CC.

\subsubsection{Mensaje de respuesta a la solicitud de medida}

El mensaje de respuesta a la solicitud de medida se enviará por parte del CC a paparazzi como respuesta al mensaje de solicitud de medida.

Tiene una longitud de $6$ bytes y la estructura en bytes especificada en la tabla \ref{tab8}.

\begin{table}[h]
	\centering
	\caption{Estructura del mensaje de respuesta a la solicitud de medida}
	\begin{tabular}{|c|c|c|c|}\hline 
		\textbf{Posición}	& \textbf{Valor} & \textbf{Tipo} &\textbf{Número de bytes} \\ \hline \hline 
		$0$		& Byte de sincronía ``R''				& ASCII	 			&	$1$ \\  \hline
		$1$		& Tipo de mensaje ``O''				& ASCII	 			&	$1$ \\  \hline
		$2-3$	& Marca de tiempo (s)				& Entero sin signo	&   $2$ \\  \hline
		$4-5$	& Checksum 							& Entero sin signo	&   $2$ \\  \hline
	\end{tabular}
	\label{tab8}
\end{table}

En la tabla \ref{tab9} puede verse un ejemplo de un mensaje de respuesta a la solicitud de medida.

\begin{table}
	\centering
	\caption{Ejemplo de respuesta a la solicitud de medida}
	\begin{tabular}{|c|c|c|c|}\hline
		\textbf{Byte} 	&	\textbf{Valor (en hexadecimal)}	&\textbf{Valor}	&\textbf{Significado} \\ \hline \hline
		$0$ 			&  $0x52$			& ``R''	& Byte sincronía	\\ \hline
		$1$				&  $0x4F$			& ``O''	& Tipo mensaje		\\ \hline
		$2$				&  $0x00$			& \multirow{2}{*}{$72$ segundos} & \multirow{2}{*}{Tiempo} \\
		$3$				&  $0x48$			&  & \\ \hline	
		$4$				&  $0x00$			&  \multirow{2}{*}{233}	& \multirow{2}{*}{checksum} \\
		$5$				&  $0xE9$			&     &     \\ \hline	
		
		
	\end{tabular}
	\label{tab9}
\end{table}

\subsubsection{Mensaje periódico con la profundidad de la sonda (durante la medida)}

El mensaje periódico con la profundidad de la sonda se enviará por parte del CC a paparazzi con una periodicidad de $1s$ mientras la sonda está midiendo.

Tiene una longitud de $8$ bytes y la estructura en bytes especificada en la tabla \ref{tab10}.

\begin{table}[h]
	\centering
	\caption{Estructura del mensaje periódico con la profundidad de la sonda}
	\begin{tabular}{|c|c|c|c|}\hline 
		\textbf{Posición}	& \textbf{Valor} & \textbf{Tipo} &\textbf{Número de bytes} \\ \hline \hline 
		$0$		& Byte de sincronía ``R''				& ASCII	 			&	$1$ \\  \hline
		$1$		& Tipo de mensaje ``M''				& ASCII	 			&	$1$ \\  \hline
		$2-3$	& Marca de tiempo (s)				& Entero sin signo	&   $2$ \\  \hline
		$4-5$	& Profundidad (mm)  				& Entero sin signo	&   $2$ \\  \hline
		$6-7$	& Checksum 							& Entero sin signo	&   $2$ \\  \hline
	\end{tabular}
\label{tab10}
\end{table}

En la tabla \ref{tab11} puede verse un ejemplo de un mensaje de profundidad de la sonda.

\begin{table}[h]
	\centering
	\caption{Ejemplo de mensaje de profundidad de la sonda}
	\begin{tabular}{|c|c|c|c|}\hline
		\textbf{Byte} 	&	\textbf{Valor (en hexadecimal)}	&\textbf{Valor}	&\textbf{Significado} \\ \hline \hline
		$0$ 			&  $0x52$			& ``R''	& Byte sincronía	\\ \hline
		$1$				&  $0x4D$			& ``M''	& Tipo mensaje		\\ \hline
		$2$				&  $0x00$			& \multirow{2}{*}{$72$ segundos} & \multirow{2}{*}{Tiempo} \\
		$3$				&  $0x48$			&  & \\ \hline	
		$4$				&  $0x3D$			& \multirow{2}{*}{$15820$ mm} & \multirow{2}{*}{Profundidad} \\
		$5$				&  $0xCC$			&  & \\ \hline	
		$6$				&  $0x01$			&  \multirow{2}{*}{496}	& \multirow{2}{*}{checksum} \\
		$7$				&  $0xF0$			&     &     \\ \hline	
		
		
	\end{tabular}
	\label{tab11}
\end{table}


\subsubsection{Mensaje de finalización de medida}

El mensaje de finalización de la medida se enviará por parte del CC a paparazzi al terminar el procedimiento de medida de la sonda.

Tiene una longitud de $8$ bytes y la estructura en bytes especificada en la tabla \ref{tab12}.

\begin{table}[h]
	\centering
	\caption{Estructura del mensaje  de finalización de medida}
	\begin{tabular}{|c|c|c|c|}\hline 
		\textbf{Posición}	& \textbf{Valor} & \textbf{Tipo} &\textbf{Número de bytes} \\ \hline \hline 
		$0$		& Byte de sincronía ``R''				& ASCII	 			&	$1$ \\  \hline
		$1$		& Tipo de mensaje ``F''				& ASCII	 			&	$1$ \\  \hline
		$2-3$	& Marca de tiempo (s)				& Entero sin signo	&   $2$ \\  \hline
		$4-5$	& Profundidad (mm)  				& Entero sin signo	&   $2$ \\  \hline
		$6-7$	& Checksum 							& Entero sin signo	&   $2$ \\  \hline
	\end{tabular}
\label{tab12}
\end{table}

En la tabla \ref{tab13} puede verse un ejemplo de un mensaje de finalización de medida.

\begin{table}[h]
	\centering
	\caption{Ejemplo de mensaje de finalización de medida}
	\begin{tabular}{|c|c|c|c|}\hline
		\textbf{Byte} 	&	\textbf{Valor (en hexadecimal)}	&\textbf{Valor}	&\textbf{Significado} \\ \hline \hline
		$0$ 			&  $0x52$			& ``R''	& Byte sincronía	\\ \hline
		$1$				&  $0x46$			& ``M''	& Tipo mensaje		\\ \hline
		$2$				&  $0x00$			& \multirow{2}{*}{$72$ segundos} & \multirow{2}{*}{Tiempo} \\
		$3$				&  $0x48$			&  & \\ \hline	
		$4$				&  $0x00$			& \multirow{2}{*}{$0$ mm} & \multirow{2}{*}{Profundidad} \\
		$5$				&  $0x00$			&  & \\ \hline	
		$6$				&  $0x00$			&  \multirow{2}{*}{224}	& \multirow{2}{*}{checksum} \\
		$7$				&  $0xE0$			&     &     \\ \hline	
		
		
	\end{tabular}
	\label{tab13}
\end{table}

\subsubsection{Mensaje de respuesta a la solicitud de posición de la sonda con la posición}

El mensaje de respuesta a la solicitud de posición de la sonda se enviará por parte del CC a paparazzi como respuesta al mensaje de solicitud de posición por parte de paparazzi.

Tiene una longitud de $8$ bytes y la estructura en bytes especificada en la tabla \ref{tab14}.

\begin{table}[h]
	\centering
	\caption{Estructura del mensaje de posición de la sonda}
	\begin{tabular}{|c|c|c|c|}\hline 
		\textbf{Posición}	& \textbf{Valor} & \textbf{Tipo} &\textbf{Número de bytes} \\ \hline \hline 
		$0$		& Byte de sincronía ``R''				& ASCII	 			&	$1$ \\  \hline
		$1$		& Tipo de mensaje ``P''				& ASCII	 			&	$1$ \\  \hline
		$2-3$	& Marca de tiempo (s)				& Entero sin signo	&   $2$ \\  \hline
		$4-5$	& Profundidad (mm)  				& Entero sin signo	&   $2$ \\  \hline
		$6-7$	& Checksum 							& Entero sin signo	&   $2$ \\  \hline
	\end{tabular}
	\label{tab14}
\end{table}

En la tabla \ref{tab15} puede verse un ejemplo de un mensaje de posición de la sonda.

\begin{table}[h]
	\centering
	\caption{Ejemplo de mensaje de posición de la sonda}
	\begin{tabular}{|c|c|c|c|}\hline
		\textbf{Byte} 	&	\textbf{Valor (en hexadecimal)}	&\textbf{Valor}	&\textbf{Significado} \\ \hline \hline
		$0$ 			&  $0x52$			& ``R''	& Byte sincronía	\\ \hline
		$1$				&  $0x50$			& ``P''		& Tipo mensaje		\\ \hline
		$2$				&  $0x00$			& \multirow{2}{*}{$72$ segundos} & \multirow{2}{*}{Tiempo} \\
		$3$				&  $0x48$			&  & \\ \hline	
		$4$				&  $0x12$			& \multirow{2}{*}{$4776$ mm} & \multirow{2}{*}{Profundidad} \\
		$5$				&  $0xA8$			&  & \\ \hline	
		$6$				&  $0x01$			&  \multirow{2}{*}{420}	& \multirow{2}{*}{checksum} \\
		$7$				&  $0xA4$			&     &     \\ \hline	
		
		
	\end{tabular}
	\label{tab15}
\end{table}


\section{Archivo de configuración}\label{sec:2}

El archivo de configuración de este módulo es \textit{conf/modules/serial\_com.xml}.

Este archivo contiene:
\begin{enumerate}
	\item La descripción de los parámetros que habrá que configurar: el puerto de conexión y la velocidad
	\begin{lstlisting}[style=XML]
	<!DOCTYPE module SYSTEM "module.dtd">
	
	<module name="serial_com" dir="com">
	<doc>
	<description>
	Decoder for serial protocol
	
	Data are extracted and sent from a serial link 
	</description>
	<configure name="SERIAL_UART" value="UARTX" description="SERIAL on which other device is connected"/>
	<configure name="SERIAL_BAUD" value="B9600" description="UART Baudrate, default to 9600"/>
	</doc>
	
	\end{lstlisting}
	
	\item Los parámetros para el envío de los mensajes a la telemetría
		\begin{lstlisting}[style=XML]
	<settings>
	<dl_settings>
	<dl_settings NAME="SR_Com">
	<dl_setting MAX="1" MIN="0" STEP="1" VAR="serial_msg_setting" shortname="stream" module="modules/com/serial_com" values="FALSE|TRUE"/>
	</dl_settings>
	</dl_settings>
	</settings>
		\end{lstlisting}
	
	\item Las dependencias. En este caso el módulo solo depende de \textit{uart}
		\begin{lstlisting}[style=XML]
	<dep>
	<depends>uart</depends>
	</dep>
		\end{lstlisting}
	
	\item Las funciones programadas en el módulo (ver \ref{sec:2}) que permiten iniciar la tarea, reaccionar a eventos y enviar periódicamente los mensajes.
		\begin{lstlisting}[style=XML]
	<header>
	<file name="serial_com.h"/>
	</header>
	<init fun="serial_init()"/>
	<periodic fun="serial_ping()" freq="10" autorun="TRUE"/>
	<event fun="serial_event()"/>
	\end{lstlisting}
	
	\item Los parámetros por defecto para el \textit{makefile}
	
	\begin{lstlisting}[style=XML]
	<makefile>
	<configure name="SERIAL_UART" default = "UART5" case="upper|lower"/>
	<configure name="SERIAL_BAUD" default="B9600"/>
	<file name="serial_com.c"/>
	<define name="USE_$(SERIAL_UART_UPPER)"/> <!-- for uart_arch-->
	<define name="SERIAL_DEV" value="$(SERIAL_UART_LOWER)"/>
	<define name="$(SERIAL_UART_UPPER)_BAUD" value="$(SERIAL_BAUD)"/>
	</makefile>
	</module>
	\end{lstlisting}

\end{enumerate}


\section{Archivos de código} \label{sec:3}

El código que gobierna el funcionamiento de la comunicación a través del puerto serie está en los archivos: \textit{serial\_com.c} y \textit{serial\_com.h}.

\subsection{Estructuras}
Se han creado dos estructuras para almacenar el mensaje que se envía y el que se recibe.

\begin{lstlisting}[language=C]
struct serial_send_t {

uint8_t msg_length;

uint8_t msgData[SERIAL_MAX_PAYLOAD];
uint8_t error_cnt;
uint8_t error_last;


uint16_t time;

int32_t lon;
int32_t lat;
int32_t alt;

uint32_t distance;
uint8_t confidence;

uint16_t ck;

};
\end{lstlisting}

\begin{lstlisting}[style=C]
struct serial_parse_t {

uint8_t msg_id;

uint8_t msgData[SERIAL_MAX_MSG] __attribute__((aligned));
uint8_t status;

uint8_t error_cnt;
uint8_t error_last;

uint8_t payload_len;

uint16_t ck;
bool msg_available;


uint16_t time;
uint16_t depth;
};
\end{lstlisting}

\subsection{Funciones auxiliares}
También una serie de funciones auxiliares que facilitan la codificación y decodificación de los mensajes:
\begin{itemize}
	\item Función que devuelve la longitud total del mensaje teniendo en cuenta los bytes de cabecera, los bytes del checksum y los bytes del contenido del mensaje.
	\begin{lstlisting}[style=C]
	static uint32_t msgLength(void);
	\end{lstlisting}
	\item Función que devuelve el valor del \textit{checksum} y lo escribe en el campo \texttt{ck} de la estructura del mensaje recibido \texttt{serial\_parse\_t}.
	\begin{lstlisting}[style=C]
	static uint32_t serial_calculateChecksum(void);
	\end{lstlisting}
	\item Función que calcula el valor del \textit{checksum} del mensaje que va a enviarse, lo codifica en $2$ bytes y lo escribe en las posiciones correspondientes del mensaje de salida. Lo escribe también en el campo \texttt{ck} de la estructura \texttt{serial\_send\_t}.
	\begin{lstlisting}[style=C]
	void serial_calculateChecksumMsg(uint8_t *msg, int msgLength);
	\end{lstlisting}
	\item Convierte una cadena de bytes de longitud \texttt{length} en un entero sin signo. Devuelve el entero sin signo. Considera codificación estilo \textit{Little Endian}.
	\begin{lstlisting}[style=C]
	unsigned int serial_byteToint(uint8_t * bytes,int length);
	\end{lstlisting}
	\item Codifica un entero sin signo en $2$ bytes.Considera codificación estilo \textit{Little Endian}.
	\begin{lstlisting}[style=C]
	void ito2h(int value, unsigned char* str); 
	\end{lstlisting}
	\item Codifica un entero con signo en una cadena de longitud \texttt{nbytes}. Si el tamaño del entero sobrepasa el número de bytes de la cadena no realiza el cálculo. El LSB corresponde al signo, si $LSB=0x01$ el número es negativo y si $LSB=0x00$ el número es postivo.
	\begin{lstlisting}[style=C]
	void itoh(int value, unsigned char* str, int nbytes);
	\end{lstlisting}
\end{itemize}

\subsection{Funciones de decodificación de los mensajes que se reciben}

Estas funciones realizan la recepción, almacenamiento y decodificación de los mensajes que llegan a paparazzi.
\begin{itemize}
	\item Esta función recibe uno a uno los bytes que se leen por el puerto serie y los va almacenando en el campo \texttt{msgData} de la estructura \texttt{serial\_parse\_t}. Comprueba que el byte inicial es correcto. Si se recibe algún byte que no concuerda con la estructura esperada del mensaje rellena el campo \texttt{error\_last} con el error \texttt{SERIAL\_ERROR\_UNEXPECTED}. Cuando se han recibido todos los bytes que completan un mensaje se indica que hay un mensaje disponible para la decodificación poniendo el campo \texttt{available} de la estructura \texttt{serial\_parse\_t} a $true$.
	\begin{lstlisting}[style=C]
	static void serial_parse(uint8_t byte);
	\end{lstlisting}
	\item Esta función decodifica el mensaje almacenado en el campo \texttt{msgData} de la estructura \texttt{serial\_parse\_t} rellenando adecuadamente los campos de datos \texttt{depth} y \texttt{time}. Para ello se ayuda de las dos funciones \texttt{message\_parse} y \texttt{message\_OK\_pase} explicadas más adelante. Comprueba que el valor del \textit{checksum} contenido en el mensaje coincide con la suma de los bytes del mensaje. De no ser así \textbf{sigue decodificando el mensaje} pero devuelve el error \texttt{SERIAL\_BR\_ERR\_CHECKSUM} en el campo \texttt{error\_last}.
	\begin{lstlisting}[style=C]
	void serial_read_message(void);
	\end{lstlisting}
	\item Decodifica el mensaje \texttt{RO} alamacenando el valor del tiempo \texttt{time}.
	\begin{lstlisting}[style=C]
	static void message_OK_parse(void);
	\end{lstlisting}
	\item Decodifica el resto de mensajes almacenando los valores \texttt{time} y \texttt{depth}.
	\begin{lstlisting}[style=C]
	static void message_parse(void);
	\end{lstlisting}
	
\end{itemize}

\subsection{Función para el envío de bytes por el puerto serie}

La siguiente función es la encargada de enviar un mensaje byte a byte por el puerto serie. Necesita un puntero al array de bytes que se quieren enviar y la longitud del array.
\begin{lstlisting}[style=C]
static void serial_send_msg(uint8_t len, uint8_t *bytes);
\end{lstlisting}

Hay también una función que se invoca periódicamente (sección \ref{sec:2}) que se encarga de enviar el mensaje seleccionado por el puerto serie. Para ello comprueba que se ha completado la recepción de un mensaje completo y selecciona con una estructura \texttt{switch-case} el mensaje que se va a enviar. Es la variable global \texttt{message\_type} la que indica el mensaje seleccionado. Dicha variable puede tomar los siguientes valores: \texttt{TELEMETRY\_SN} si se quiere enviar un mensaje de telemetría (ver \ref{sec:1}), \texttt{SONDA\_RQ} si se quiere solicitar la profundidad a la que se encuentra la sonda ( ver \ref{sec:1}) o \texttt{MEASURE\_SN} si se quiere indicar al CC que comience a bajar la sonda para medir (ver \ref{sec:1}). En el caso de que la variable \texttt{message\_type} tenga un valor distinto de los tres anteriores se indicará con un error \texttt{SERIAL\_ERR\_UNEXPECTED}. 

Esta función construye en mensaje que se quiere enviar con los bytes adecuados e invoca a la función \texttt{serial\_send\_message}. Para ello obtiene la posición del GPS  (en coordenadas geodésicas) invocando la función \texttt{stateGetPositionLla\_i()}y los datos del sonar mediante la función \texttt{sonar\_get()} y los codifica en bytes según el convenio

\begin{lstlisting}[style=C]
void serial_ping(void);
\end{lstlisting}


\subsection{Función para la recepción de bytes por el puerto serie}

La siguiente función se invoca cuando se produce un evento asociado al puerto serie. En ese caso se comprueba si hay un caracter disponible y de ser así se obtiene y se envía al decoficador de bytes \texttt{serial\_parse}. Si tras decodificar ese byte el mensaje está completo se invoca a la función de decodificación del mensaje \texttt{serial\_read\_message}.
 \begin{lstlisting}[style=C]
 void serial_event(void);
 \end{lstlisting}
 
 \subsection{Otras funciones}
 
 Finalmente tenemos otras dos funciones:
 \begin{itemize}
 	\item Función para inicializar la comunicación serie. Esta función ha de invocarse en el archivo de configuración como función de \texttt{init} (ver \ref{sec:2}).
 	\begin{lstlisting}[style=C]
 	void serial_init(void);
 	\end{lstlisting}
 	\item Función para enviar los datos por telemtría. Esta función hace uso del mensaje \texttt{SERIAL\_COM} de la telemetría para mostrar en el interfaz gráfico tanto los mensajes que se envían como los mensajes que se reciben.
 	\begin{lstlisting}[style=C]
 	static void send_telemetry(struct transport_tx *trans, struct link_device *dev);
 	\end{lstlisting}
 \end{itemize}
 
 
 