Esto es un simple guía burros para paparazzi, dado que entender lo que dice es complicadillo.
En principio, podemos dividirlo en capítulos, y emplear para cada uno un archivo nuevo.

De todos modos, lo importante es ir escribiendo y luego vemos como organizarlo. \\

Antes de comenzar, indicar que a lo largo de todo este documento vamos a trabajar con Linux, más concretamente con el sistema operativo Ubuntu en su versión 20.04. Este es el único entorno que va a soportar PaparazziUAV.\\

\noindent El equipo que desarrolla PaparazziUAV cuenta con una documentación bastante detallada que puede llegar a ser de gran utilidad para incorporarse al entorno: \url{https://paparazzi-uav.readthedocs.io/en/stable/index.html}. Para instalar PaparazziUAV recomendamos al usuario seguir directamente la entrada \textit{installation} (\url{https://paparazzi-uav.readthedocs.io/en/stable/installation/index_installation.html}) y leer el \textit{README.md} principal del proyecto (\url{https://github.com/UCM-237/paparazzi}). \\
